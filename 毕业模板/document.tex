% 导言区
%导入article类   还有book,report,letter其他类
%主要用于进行全局设置
%\documentclass{article}   %book report letter article

\documentclass{ctexart}  %ctexart等价于使用article类 然后再导入ctex宏包

%\usepackage{ctex} %引入宏包  texdoc ctex查看说明文档


\title{\heiti 勾股定理}  %文档标题 黑体
\author{\kaishu Berlin} %作者姓名 楷书
\date{\today}  %文档时间今天


%正文区(文稿区)
\begin{document}
	\maketitle  %输出标题 
	%section subsection subsubsection 等命令可以设置提纲
	\section{引言}
	%插入一段正文 空行代表另起一段有缩紧(多个空行相当于一个);\\代表换行没有缩紧;\par相当于空行另起一段 一般为了结构清晰 更多使用空行
	有这么一句话在业界广泛流传:数据和特征决定了机器学习的上限,而模型和算法只是逼近这个上限而已。那特征工程到底是什么呢?顾名思义,其本质是一项工程活动,目的是最大限度地从原始数据中提取特征以供算法和模型使用。通过总结和归纳,目前认为特征工程包括以下方面。
	
	有这么一句话在业界广泛流传:数据和特征决定了机器学习的上限,而模型和算法只是逼近这个上限而已。那特征工程到底是什么呢?\\顾名思义,其本质是一项工程活动,目的是最大限度地从原始数据中提取特征以供算法和模型使用。通过总结和归纳,\par 目前认为特征工程包括以下方面。
	\section{实验方法}
	
	\section{实验结果}
	
	\subsection{数据}
	\subsection{图表}
	\subsubsection{实验条件}
	\subsubsection{实验过程}
	\subsection{结果分析}
	\section{结论}
	\section{致谢}
	

	%\begin{tabular}[<垂直对齐方式>]{<列格式说明>}
	%	<表项> & <表项> & <表项> & ...&<表项>\\
	\begin{tabular}{l||c|c|c|r|p{1.5cm}}
		\hline \hline
		姓名 & 语文 & 数学 & 英语 & 物理 & 备注\\
		\hline \hline
		张三 & 87 &100 & 90 & 100 & 良好\\
		\hline
		李四 & 20 & 30 & 40 & 50 & 补考,请等待通知。\\
		\hline
		张无忌 & 100 & 100 & 100 & 100 & 优秀\\
	\end{tabular}
	
	
\end{document}